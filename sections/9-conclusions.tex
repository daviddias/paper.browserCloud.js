%!TEX root = ../index.tex

\section{Conclusions}

We end this report by making an overview and summing up all the primary aspects, from proposed work, contributions, state of the art, definition of the architecture, implementation of the respective architecture and evaluation, moving to what were the major breakthroughs and ending with concluding remarks and future work.

browserCloud.js was an exercise to create a distributed computing fabric, enabling its users to share effectively their resources, giving developers a reliable and efficient way to store and process data for their applications.

The system was designed to be native to the Web Platform, the most ubiquitous runtime. There were two reasons behind this decision, the first being longevity, the Web Platform, even though it is quite popular, it is still an emerging platform, meaning that our assumptions of ubiquity will previal; the second reason was developer adoption, JavaScript is the "lingua franca" of the web, meaning that it will be common for a developer to know how to code with JavaScripts APIs and since browserCloudjs was built in JavaScript, developers will know inherently how to use the platform.

Going after a decentralized model was also something we saw as a potential key factor for the browserCloudjs platform, structured peer-2-peer networks scale well with demand, while centralized networks have a number of significant challenges once a certain threshold of users is reached. WebRTC, the technology enabling browsers to communicate in a peer-2-peer way, is in great part responsible for this platform success.

With browserCloudjs, we achieved in bulk, mainly two great milestones:

\begin{itemize}
  \item \textbf{The first browser based DHT} - browserCloudjs offers for the first time in browser history a fully functional DHT, performing resource decentralized resource discovery on the browser. 
  \item \textbf{The first peer-2-peer browser computing platform} - the research of using browsers to leverage the idle computer cycles have been in the literature for a while, however, always following the centralized/BOINC model. browserCloudjs offers the first peer-2-peer browser computing framework with proven speedups.
\end{itemize}

We have found this paper to be a source of hard work and enthusiasm, a great opportunity to research and interact with bleeding edge technologies and also, interact with the developer communities that are pushing the web forward.
