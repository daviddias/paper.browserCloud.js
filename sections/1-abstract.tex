%!TEX root = ../index.tex

% 
% Abstract 
% 

\begin{abstract}

Grid Computing has been around since the 90's, its fundamental basis is to use idle resources in geographically distributed systems in order to maximize their efficiency, giving researchers access to computational resources to perform their jobs (e.g. studies, simulations, rendering, data processing, etc). This approach quickly grew into non Grid environments, causing the appearance of projects such as SETI@Home or Folding@Home, leveraging volunteered shared resources and not only institution-wide data centers as before, giving the birth of Public Computing. Today, after having volunteering computing as a proven concept, we face the challenge of how to create a simple, effective, way for people to participate in such community efforts and even more importantly, how to reduce the friction of adoption by the developers and researchers to use and provide these resources for their applications. This thesis explores and proposes novel ways to enable end user machines to communicate, using recent Web technologies such as WebRTC, creating a simple API that is familiar to those used to develop applications for the Cloud, but with resources provided by a community and not by a company or institution.

\end{abstract}
